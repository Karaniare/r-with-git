% Options for packages loaded elsewhere
\PassOptionsToPackage{unicode}{hyperref}
\PassOptionsToPackage{hyphens}{url}
%
\documentclass[
  ignorenonframetext,
]{beamer}
\usepackage{pgfpages}
\setbeamertemplate{caption}[numbered]
\setbeamertemplate{caption label separator}{: }
\setbeamercolor{caption name}{fg=normal text.fg}
\beamertemplatenavigationsymbolsempty
% Prevent slide breaks in the middle of a paragraph
\widowpenalties 1 10000
\raggedbottom
\setbeamertemplate{part page}{
  \centering
  \begin{beamercolorbox}[sep=16pt,center]{part title}
    \usebeamerfont{part title}\insertpart\par
  \end{beamercolorbox}
}
\setbeamertemplate{section page}{
  \centering
  \begin{beamercolorbox}[sep=12pt,center]{part title}
    \usebeamerfont{section title}\insertsection\par
  \end{beamercolorbox}
}
\setbeamertemplate{subsection page}{
  \centering
  \begin{beamercolorbox}[sep=8pt,center]{part title}
    \usebeamerfont{subsection title}\insertsubsection\par
  \end{beamercolorbox}
}
\AtBeginPart{
  \frame{\partpage}
}
\AtBeginSection{
  \ifbibliography
  \else
    \frame{\sectionpage}
  \fi
}
\AtBeginSubsection{
  \frame{\subsectionpage}
}
\usepackage{amsmath,amssymb}
\usepackage{lmodern}
\usepackage{ifxetex,ifluatex}
\ifnum 0\ifxetex 1\fi\ifluatex 1\fi=0 % if pdftex
  \usepackage[T1]{fontenc}
  \usepackage[utf8]{inputenc}
  \usepackage{textcomp} % provide euro and other symbols
\else % if luatex or xetex
  \usepackage{unicode-math}
  \defaultfontfeatures{Scale=MatchLowercase}
  \defaultfontfeatures[\rmfamily]{Ligatures=TeX,Scale=1}
\fi
\usetheme[]{CambridgeUS}
\usecolortheme{beaver}
% Use upquote if available, for straight quotes in verbatim environments
\IfFileExists{upquote.sty}{\usepackage{upquote}}{}
\IfFileExists{microtype.sty}{% use microtype if available
  \usepackage[]{microtype}
  \UseMicrotypeSet[protrusion]{basicmath} % disable protrusion for tt fonts
}{}
\makeatletter
\@ifundefined{KOMAClassName}{% if non-KOMA class
  \IfFileExists{parskip.sty}{%
    \usepackage{parskip}
  }{% else
    \setlength{\parindent}{0pt}
    \setlength{\parskip}{6pt plus 2pt minus 1pt}}
}{% if KOMA class
  \KOMAoptions{parskip=half}}
\makeatother
\usepackage{xcolor}
\IfFileExists{xurl.sty}{\usepackage{xurl}}{} % add URL line breaks if available
\IfFileExists{bookmark.sty}{\usepackage{bookmark}}{\usepackage{hyperref}}
\hypersetup{
  pdftitle={R With Git and Github},
  pdfauthor={Aris Paschalidis},
  hidelinks,
  pdfcreator={LaTeX via pandoc}}
\urlstyle{same} % disable monospaced font for URLs
\newif\ifbibliography
\usepackage{color}
\usepackage{fancyvrb}
\newcommand{\VerbBar}{|}
\newcommand{\VERB}{\Verb[commandchars=\\\{\}]}
\DefineVerbatimEnvironment{Highlighting}{Verbatim}{commandchars=\\\{\}}
% Add ',fontsize=\small' for more characters per line
\usepackage{framed}
\definecolor{shadecolor}{RGB}{248,248,248}
\newenvironment{Shaded}{\begin{snugshade}}{\end{snugshade}}
\newcommand{\AlertTok}[1]{\textcolor[rgb]{0.94,0.16,0.16}{#1}}
\newcommand{\AnnotationTok}[1]{\textcolor[rgb]{0.56,0.35,0.01}{\textbf{\textit{#1}}}}
\newcommand{\AttributeTok}[1]{\textcolor[rgb]{0.77,0.63,0.00}{#1}}
\newcommand{\BaseNTok}[1]{\textcolor[rgb]{0.00,0.00,0.81}{#1}}
\newcommand{\BuiltInTok}[1]{#1}
\newcommand{\CharTok}[1]{\textcolor[rgb]{0.31,0.60,0.02}{#1}}
\newcommand{\CommentTok}[1]{\textcolor[rgb]{0.56,0.35,0.01}{\textit{#1}}}
\newcommand{\CommentVarTok}[1]{\textcolor[rgb]{0.56,0.35,0.01}{\textbf{\textit{#1}}}}
\newcommand{\ConstantTok}[1]{\textcolor[rgb]{0.00,0.00,0.00}{#1}}
\newcommand{\ControlFlowTok}[1]{\textcolor[rgb]{0.13,0.29,0.53}{\textbf{#1}}}
\newcommand{\DataTypeTok}[1]{\textcolor[rgb]{0.13,0.29,0.53}{#1}}
\newcommand{\DecValTok}[1]{\textcolor[rgb]{0.00,0.00,0.81}{#1}}
\newcommand{\DocumentationTok}[1]{\textcolor[rgb]{0.56,0.35,0.01}{\textbf{\textit{#1}}}}
\newcommand{\ErrorTok}[1]{\textcolor[rgb]{0.64,0.00,0.00}{\textbf{#1}}}
\newcommand{\ExtensionTok}[1]{#1}
\newcommand{\FloatTok}[1]{\textcolor[rgb]{0.00,0.00,0.81}{#1}}
\newcommand{\FunctionTok}[1]{\textcolor[rgb]{0.00,0.00,0.00}{#1}}
\newcommand{\ImportTok}[1]{#1}
\newcommand{\InformationTok}[1]{\textcolor[rgb]{0.56,0.35,0.01}{\textbf{\textit{#1}}}}
\newcommand{\KeywordTok}[1]{\textcolor[rgb]{0.13,0.29,0.53}{\textbf{#1}}}
\newcommand{\NormalTok}[1]{#1}
\newcommand{\OperatorTok}[1]{\textcolor[rgb]{0.81,0.36,0.00}{\textbf{#1}}}
\newcommand{\OtherTok}[1]{\textcolor[rgb]{0.56,0.35,0.01}{#1}}
\newcommand{\PreprocessorTok}[1]{\textcolor[rgb]{0.56,0.35,0.01}{\textit{#1}}}
\newcommand{\RegionMarkerTok}[1]{#1}
\newcommand{\SpecialCharTok}[1]{\textcolor[rgb]{0.00,0.00,0.00}{#1}}
\newcommand{\SpecialStringTok}[1]{\textcolor[rgb]{0.31,0.60,0.02}{#1}}
\newcommand{\StringTok}[1]{\textcolor[rgb]{0.31,0.60,0.02}{#1}}
\newcommand{\VariableTok}[1]{\textcolor[rgb]{0.00,0.00,0.00}{#1}}
\newcommand{\VerbatimStringTok}[1]{\textcolor[rgb]{0.31,0.60,0.02}{#1}}
\newcommand{\WarningTok}[1]{\textcolor[rgb]{0.56,0.35,0.01}{\textbf{\textit{#1}}}}
\setlength{\emergencystretch}{3em} % prevent overfull lines
\providecommand{\tightlist}{%
  \setlength{\itemsep}{0pt}\setlength{\parskip}{0pt}}
\setcounter{secnumdepth}{-\maxdimen} % remove section numbering
\ifluatex
  \usepackage{selnolig}  % disable illegal ligatures
\fi

\title{R With Git and Github}
\author{Aris Paschalidis}
\date{September 02, 2021}

\begin{document}
\frame{\titlepage}

\begin{frame}[allowframebreaks]
  \tableofcontents[hideallsubsections]
\end{frame}
\hypertarget{setup}{%
\section{Setup}\label{setup}}

\begin{frame}[fragile]{Install Git}
\protect\hypertarget{install-git}{}
Git allows us to track changes to our documents (i.e.~Git is version
control).

\begin{itemize}
\item
  Check if git is already installed: \texttt{which\ git}
\item
  Install git for Windows: \url{https://gitforwindows.org/}
\item
  Install git for Mac: \texttt{brew\ install\ git}
\item
  Install git for Linux:

  \begin{itemize}
  \tightlist
  \item
    Ubuntu or Debian: \texttt{sudo\ apt-get\ install\ git}
  \item
    Fedora or RedHat: \texttt{sudo\ yum\ install\ git}
  \end{itemize}
\end{itemize}
\end{frame}

\begin{frame}[fragile]{Configure Git}
\protect\hypertarget{configure-git}{}
\begin{itemize}
\tightlist
\item
  Introduce yourself to git
\end{itemize}

\begin{Shaded}
\begin{Highlighting}[]
\NormalTok{usethis}\SpecialCharTok{::}\FunctionTok{use\_git\_config}\NormalTok{(}\AttributeTok{user.name =} \StringTok{"first last"}\NormalTok{,}
                        \AttributeTok{user.email =} \StringTok{"github email"}\NormalTok{)}
\end{Highlighting}
\end{Shaded}

\begin{itemize}
\tightlist
\item
  Ensure setup was successful
\end{itemize}

\begin{Shaded}
\begin{Highlighting}[]
\FunctionTok{git}\NormalTok{ config }\AttributeTok{{-}{-}global} \AttributeTok{{-}{-}list} 
\end{Highlighting}
\end{Shaded}
\end{frame}

\begin{frame}[fragile]{Managing Git(Hub) Credentials\footnote<.->{Following
  the
  \href{https://usethis.r-lib.org/articles/articles/git-credentials.html}{usethis
  guide}}}
\protect\hypertarget{managing-github-credentialsref}{}
\begin{itemize}
\tightlist
\item
  Adopt HTTPS
\item
  \href{https://docs.github.com/en/github/authenticating-to-github/securing-your-account-with-two-factor-authentication-2fa}{Secure
  your account with 2FA}
\item
  Create a PAT: \texttt{usethis::create\_github\_token()}
\item
  Store PAT into the Git credential store:
  \texttt{gitcreds::gitcreds\_set()}
\end{itemize}
\end{frame}

\hypertarget{git-basics}{%
\section{Git Basics}\label{git-basics}}

\begin{frame}{Stage \& Commit}
\protect\hypertarget{stage-commit}{}
\begin{itemize}
\tightlist
\item
  We can create and edit files within our project
\item
  We can then \textbf{stage} and \textbf{commit} these changes
\end{itemize}

\begin{figure}

{\centering \includegraphics[width=0.75\linewidth]{images/git-workflow} 

}

\caption{Git Workflow}\label{fig:unnamed-chunk-4}
\end{figure}
\end{frame}

\begin{frame}{Diff}
\protect\hypertarget{diff}{}
\begin{itemize}
\tightlist
\item
  We can look at the set of \emph{differences} between files to see what
  has changed
\end{itemize}

\begin{figure}

{\centering \includegraphics[width=0.6\linewidth]{images/example-diff} 

}

\caption{Example Diff}\label{fig:unnamed-chunk-5}
\end{figure}
\end{frame}

\begin{frame}{Commit Best Practices}
\protect\hypertarget{commit-best-practices}{}
Each commit should be:

\begin{itemize}
\tightlist
\item
  \emph{Minimal:} A commit should only contain changes related to a
  single problem or feature
\item
  \emph{Complete:} A commit should add the functionality it claims to
  add
\item
  \emph{Concise, yet evocative:} At a glance, you should be able to
  understand what a commit does, but you should include enough detail so
  you can remember what was done
\end{itemize}
\end{frame}

\begin{frame}{Time Travel}
\protect\hypertarget{time-travel}{}
\begin{itemize}
\tightlist
\item
  Can look at old commits and access old code
\item
  Can use your git client or Github
\item
  Can revert back to a previous commit
\end{itemize}
\end{frame}

\begin{frame}{Pull \& Push Changes}
\protect\hypertarget{pull-push-changes}{}
\begin{itemize}
\tightlist
\item
  \textbf{Pulling} changes from Github allows us to access stored
  changes
\item
  \textbf{Pushing} changes stores changes on Github
\end{itemize}

\begin{figure}

{\centering \includegraphics[width=0.6\linewidth]{images/push-pull} 

}

\caption{Accessing the Cloud}\label{fig:unnamed-chunk-6}
\end{figure}
\end{frame}

\hypertarget{exercise-use-git}{%
\section{Exercise: Use Git}\label{exercise-use-git}}

\begin{frame}[fragile]{Make a RStudio Project and Connect to Github}
\protect\hypertarget{make-a-rstudio-project-and-connect-to-github}{}
\begin{itemize}
\tightlist
\item
  Create a project: \texttt{usethis::create\_project(path)}
\item
  Use git: \texttt{usethis::use\_git()}
\item
  Connect a Github repo: \texttt{usethis::use\_github()}
\end{itemize}
\end{frame}

\begin{frame}{Commit \& Push}
\protect\hypertarget{commit-push}{}
\begin{itemize}
\tightlist
\item
  Make a .Rmd file and edit it
\item
  Commit your change
\item
  Make another change and commit it
\item
  Push your changes
\end{itemize}
\end{frame}

\hypertarget{working-with-others}{%
\section{Working With Others}\label{working-with-others}}

\begin{frame}{Overview}
\protect\hypertarget{overview}{}
\begin{figure}

{\centering \includegraphics[width=0.6\linewidth]{images/multiple-users} 

}

\caption{Multiple Developers}\label{fig:unnamed-chunk-7}
\end{figure}
\end{frame}

\begin{frame}{Fork \& Clone}
\protect\hypertarget{fork-clone}{}
\begin{itemize}
\tightlist
\item
  Developer A makes a repo
\item
  Developer B wants to make some changes
\item
  Developed B \textbf{forks and clones} the repo:
\end{itemize}

\begin{figure}

{\centering \includegraphics[width=0.6\linewidth]{images/fork-and-clone} 

}

\caption{Fork and Clone}\label{fig:unnamed-chunk-8}
\end{figure}
\end{frame}

\begin{frame}{Branches}
\protect\hypertarget{branches}{}
\begin{itemize}
\tightlist
\item
  Developer B makes a new \textbf{branch} to develop a new feature
\item
  Allows Developer A to continue working on the main stream of
  development
\item
  Developers A and B can work in parallel
\end{itemize}

\begin{figure}

{\centering \includegraphics[width=0.6\linewidth]{images/branches} 

}

\caption{Branches}\label{fig:unnamed-chunk-9}
\end{figure}
\end{frame}

\begin{frame}{Pull Requests}
\protect\hypertarget{pull-requests}{}
\begin{itemize}
\tightlist
\item
  Developer B initiates a \textbf{pull request (PR) } asking Developer A
  to incorporate changes
\item
  Developer A reviews the \textbf{PR}
\item
  Developer A makes or requests changes
\item
  Developer A merges the \textbf{PR}
\end{itemize}

\begin{figure}

{\centering \includegraphics[width=0.6\linewidth]{images/pull-request} 

}

\caption{Pull Request Workflow}\label{fig:unnamed-chunk-10}
\end{figure}
\end{frame}

\begin{frame}[fragile]{In Practice}
\protect\hypertarget{in-practice}{}
Contributor:

\begin{itemize}
\tightlist
\item
  Fork and clone: \texttt{usethis::create\_from\_github("OWNER/REPO")}
\item
  Create a branch: \texttt{usethis::pr\_init(branch)}
\item
  Push changes: \texttt{usethis::pr\_push()}
\end{itemize}

Developer:

\begin{itemize}
\tightlist
\item
  Download a PR: \texttt{usethis::pr\_fetch()}
\item
  Push changes to branch: \texttt{usethis::pr\_push()}
\item
  Merge the PR on Github
\item
  Delete the branch: \texttt{usethis::pr\_finish()}
\end{itemize}
\end{frame}

\hypertarget{exercise-create-a-pull-request}{%
\section{Exercise: Create a Pull
Request}\label{exercise-create-a-pull-request}}

\begin{frame}[fragile]{Exercise: Create a Pull Request}
\begin{itemize}
\tightlist
\item
  Fork and clone this repo:
  \texttt{usethis::create\_from\_github("arisp99/r-with-git")}
\item
  Create a branch: \texttt{usethis::pr\_init("USERNAME-fun-fact")}
\item
  Add a geography fun fact and picture to the presentation!
\item
  Push your changes: \texttt{usethis::pr\_push()}
\end{itemize}
\end{frame}

\hypertarget{geography-fun-facts}{%
\section{Geography Fun Facts!}\label{geography-fun-facts}}

\begin{frame}{Greece}
\protect\hypertarget{greece}{}
\begin{itemize}
\tightlist
\item
  The world's oldest weather station is at the base of the Acropolis
\end{itemize}

\begin{figure}

{\centering \includegraphics[width=0.6\linewidth]{images/greece} 

}

\caption{The Acropolis}\label{fig:unnamed-chunk-11}
\end{figure}
\end{frame}

\begin{frame}{Mali}
\protect\hypertarget{mali}{}
\begin{itemize}
\tightlist
\item
  The country of the righest man in history (Mansa Musa)!
\end{itemize}
\end{frame}

\hypertarget{section}{%
\section{}\label{section}}

\end{document}
